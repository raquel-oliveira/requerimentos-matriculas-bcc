\newcommand{\header}{\small\MakeUppercase{\university}\\\MakeUppercase{\area}\\\textcolor{blueDimap}{\MakeUppercase{\departament}}\\\MakeUppercase{Bacharelado em \course}
\textcolor{blueDimap}{\rule{0.8\textwidth}{4pt}}}

\newcommand{\printheader}{\fancyhead{}\renewcommand{\headrulewidth}{0pt}\rhead{\textbf{\header}}\lhead{\includegraphics[width=2cm]{Logotipo-DIMAp}}}


\newcommand{\demands}{
\noindent Caso o requerente não cumpra as exigências estipuladas no Artigo 3o, II da Resolução no 03/2016-CCCC de 02 de dezembro de 2016, conforme estipulado pelo Parágrafo Único, II do mesmo artigo, é necessário um parecer do orientador apresentando ciência das disciplinas que o aluno está matriculado e portando informações sobre o \textbf{andamento da monografia} e os \textbf{resultados já alcançados}.
\\
\\
Caso o requerente cumpra as exigências da Resolução no 03/2016-CCCC de 02 de dezembro de 2016, apenas a assinatura do orientador é suficiente.
}

%Maybe change to environment
\newcommand{\signatures}[2]
{\begin{multicols}{2}
		\begin{tabular}{|p{0.4\textwidth}|}
			\hline\\
			Assinatura #1
			\\
			\\
			\\\\\hline
		\end{tabular}
	\columnbreak 
	\begin{tabular}{|p{0.4\textwidth}|}
		\hline\\
		Assinatura #2
		\\
		\\
		\\\\\hline
	\end{tabular} 
	\end{multicols}
}

%%%%%%%
\makeatletter

\newcommand*\dotcolumnfill{%
	\par
	\null
	\vskip -\ht\strutbox
	\xleaders \hb@xt@ \hsize {%
		\strut \cleaders \hb@xt@ .44em{\hss.\hss}\hfill
	}\vfill
	\vskip \ht\strutbox
	%\break
}

\makeatother

%%%% 
\newcommand{\printStudent}[1]{\MakeUppercase{Alun#1}: \dotuline{\student\hfill}} 

\newcommand{\printStudentWhId}[1]{\MakeUppercase{Alun#1}: \dotuline{\student\hfill}  MATRICULA: \dotuline{\studentid\hfill} } 

\newcommand{\printAdvisor}[1]{\MakeUppercase{ORIENTADOR#1}: \dotuline{\advisorName \hfill}} 

\newcommand{\printAdvisorFull}[1]{\MakeUppercase{ORIENTADOR#1}: \dotuline{\advisorName\hfill}GRAU DE FORMAÇÃO:  \dotuline{\advisorDegree\hfill}  \\
CARGO: \dotuline{\advisorPosition\hfill}LOTAÇÃO: \dotuline{\advisorLotacao\hfill}} 

\newcommand{\printCoadvisor}[1]{\MakeUppercase{CO-ORIENTADOR#1}: \dotuline{\coadvisorName\hfill}} 

\newcommand{\printCoadvisorFull}[1]{\MakeUppercase{CO-ORIENTADOR#1}: \dotuline{\coadvisorName\hfill} \MakeUppercase{GRAU DE FORMAÇÃO}: \dotuline{\coadvisorDegree\hfill}\\\MakeUppercase{CARGO}: \dotuline{\coadvisorPosition\hfill} \MakeUppercase{LOTAÇÃO}: \dotuline{\coadvisorLotacao\hfill}} 

\newcommand{\printThesisName}{\MakeUppercase{Tema do trabalho}: \dotuline{\thesisTitle\hfill}} 

\newcommand{\printSemester}{\MakeUppercase{Semestre}: \dotuline{\semester\hfill}}

\newcommand{\printKeyWords}{\MakeUppercase{Palavras-chave: }: \thesisKeywords \dotuline{\hfill} }

\newcommand{\printResume}{\MakeUppercase{Resumo }:  \thesisResume \dotuline{\hfill}}